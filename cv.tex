\documentclass[english,a4paper]{europasscv}
\ecvname{Rúben Marques}
\ecvaddress{Estrada dos Paulinos CCI-1812 2870-522 Sarilhos Grandes, Portugal}
\ecvtelephone{(+351) 212 898 212}
\ecvmobile{(+351) 934 962 739}
\ecvemail{rubenmarques91@gmail.com}
\ecvhomepage{https://rmarques.io}
\ecvgithubpage{https://github.com/ras-marques https://gitlab.com/ras-marques}
\ecvlinkedinpage{https://www.linkedin.com/in/ruben-marques-6179aa60/}
\ecvim{Skype}{ras.marques}
\ecvim{Discord}{Rúben Marques\#4748}
\ecvdateofbirth{16 June 1991}
\ecvnationality{Portuguese}
\ecvgender{Male}
\ecvpictureleft[width=3.8cm]{my_photo.png}

\usepackage[numbers]{natbib}
\usepackage{bibentry}
\usepackage{notoccite}
\usepackage{textcomp}

\begin{document}
\begin{europasscv}
\ecvpersonalinfo

\ecvsection[2.5mm]{Professional Experience} 

\ecvtitle[2.5mm]{April 2017 - Present}{Co-founder and CTO}
\ecvitem{}{SurgeonMate}
\ecvitem{}{Project started in the end of 2015, being full time in the pointed duration. This initiative aims to create a smartglasses device to record surgeries and the accompaning multimedia library software to catalog and store those recordings.
\par \textbullet\,\,Participation on planning, particularly on the technical side.
\par \textbullet\,\,Firmware development, particularly on Python using a Raspberry Pi, regarding the smartglasses hardware.
\par \textbullet\,\,PCB development using EAGLE6.5.0.
\par \textbullet\,\,CNC PCB milling.
\par \textbullet\,\,Software development using javascript, regarding the multimedia library.}

\ecvtitle[2.5mm]{August 2015 - April 2017}{IT Support}
\ecvitem{}{Instituto de Plasmas e Fusão Nuclear}
\ecvitem{}{Instituto Superior Técnico}
\ecvitem{}{\textbullet\,\,IT support and Linux systems and networks admin.
\par \textbullet\,\,Regular skills on automating tasks using \textit{Python}.
\par \textbullet\,\,Regular skills on building sites using \textit{html}, \textit{php} and \textit{mysql}. 
\par \textbullet\,\,Configuration of automatic backups using \textit{bacula}.
\par \textbullet\,\,Network analisys and \textit{firewall} administration.
\par \textbullet\,\,\textit{Greylists} administration for automatic e-mail processing.}

\ecvtitle[2.5mm]{September 2014 - June 2015\\September 2013 - January 2014}{Lab activities monitor}
\ecvitem{}{Departamento de Física}
\ecvitem{}{Instituto Superior Técnico}
\ecvitem{}{4 hours per week of theoric and practical support to groups of 16-18 students in the lab activities of the following subjects:
\par \textbullet\,\,Electronic Instrumentation (2014/2015)
\par \textbullet\,\,Data Acquisition Systems (2014/2015)
\par \textbullet\,\,Microcontrollers (2013/2014)
\par Structured and direct communication in portuguese, oriented to teaching complex concepts to students with basic or no previous knowledge in the subject.}

\newpage

\ecvtitle{November 2012 - August 2013}{Scientific Initiation Grant}
\ecvitem{}{Instituto de Plasmas e Fusão Nuclear}
\ecvitem{}{Instituto Superior Técnico}
\ecvitem{}{Participation in the \textit{e-lab} project: A remote real physics experiments platform. Voluntary participation after Grant end until April 2017.
\par \textbullet\,\,Development of experimental setups and respective instrumentation and automation using dsPIC microcontrollers and integration of said experiments in the platform using \textit{Java} and \textit{XML}.
\par \textbullet\,\,2 co-authorships in one conference \textit{Proceedings} \cite{pub1} \cite{pub2} and an unpublished poster on another.
\par \textbullet\,\,Regular skills on english communications, attained through presentations and guided visits to the remote labs with groups of 20-22 international students. Participated on these events 2 times per year between March 2012 and March 2016.}

\ecvsection{Education}
%[Add separate entries for each course. Start from the most recent.]

\ecvtitle{October 2018 - March 2019}{Eddisrupt full-stack development Bootcamp}
\ecvitem{}{WYgroup, Oeiras, Portugal}
\ecvtitle{September 2012 - July 2016}{Master of Science in Engineering Physics and Technology}
\ecvitem{}{Instituto Superior Técnico de Lisboa, Portugal}
\ecvitem{}{15/20}
\ecvtitle{Setember 2009 - July 2015}{Bachelor of Science in Engineering Physics and Technology}
\ecvitem{}{Instituto Superior Técnico de Lisboa, Portugal}
\ecvitem{}{13/20}
\ecvtitle{Setember 2006 - July 2009}{High school}
\ecvitem{}{Escola Secundária da Moita, Portugal}
\ecvitem{}{18/20}

\ecvsection{Personal skills}

\ecvmothertongue[20pt]{Portuguese}
\ecvlanguageheader
\ecvlanguage{Inglês}{C2}{C2}{C2}{C2}{C2}
\ecvlanguagefooter[10pt]

\ecvblueitem{Electronics and programming\\for embedded systems}{On March 2018, took care of prototyping and building the electronics needed for a mat with pressable buttons and programmed a Raspberry Pi to display images and play music as users stepped on the correct buttons, as part of a publicitary event by \textit{Mimosa} named \textit{35\textdegree Passeio em Família}.
\par His Master Thesis focused on the design and prototyping of an electronic inverter for the control of Dahlander type induction motors in the scope of electric vehicles. Furthermore, while undergoing studies, other activities that sharpened his skills in electronics and programming included:\par
\textbullet\,\,Since September 2015 he's part of the development team of \href{http://www.surgeonmate.com/}{\textcolor{blue}{\mbox{SurgeonMate}}}. He works with Raspberry Pi, cameras, microphones, motorization and graphical user interfaces.
\par\textbullet\,\,Since July 2014 he's part of the development team of a smart asthma pump, that started as a Master Thesis of Joana Belo, a student of Faculdade de Arquitectura de Lisboa. He's responsible for the selection of adequate sensors and advises on the structural and functional aspects of the prototype.
\par \textbullet\,\,From November 2015 to March 2016 he was part of the stutents' project Técnico Solar Boat, where he developed a MATLAB program to simulate the energy use of an electric solar boat in a track with configurable parameters.
\par \textbullet\,\,The study of the behavior, the dimensioning and construction of a Tesla coil in the scope of the Introdution to Investigation course in the 2013/2014 school year.\par
\textbullet\,\,He built and programmed a three axis CNC mill from scratch as a hobby to use on future projects. The machine microcontroller is a dsPIC30F4011 and the graphical interface is programmed in Python. Code is open-source for both ends and is available on GitHub at \href{https://github.com/ras-marques/myCNC_uC}{\textcolor{blue}{\mbox{myCNC\_uC}}} and \href{https://github.com/ras-marques/myCNC_GUI}{\textcolor{blue}{\mbox{myCNC\_GUI}}}. Work is still undergoing on incremental patches to the user interface and machine functions.
\par \vspace{3mm}
These activities, together with his Master Thesis and professional experience amounted to:\par
\textbullet\,\,+1500h of C programming applied to PIC microcontrollers
\par\textbullet\,\,$\sim$400h of C programming applied to Arduino microcontrollers
\par\textbullet\,\,$\sim$300h of Python programming applied to system administration automation, firmware and graphical user interfaces.
\par\textbullet\,\,$\sim$300h of development of control and protection circuits for power electronics.
\par\textbullet\,\,$\sim$300h of development of amplification and filtering circuits for analog and digital signals.
\par\textbullet\,\,Regular skills using ORCAD 9.2 and EAGLE 6.5.0 for designing schematics and
layouts of Printed Circuit Boards.
\par\textbullet\,\,Regular skills using MATLAB for physics simulation.
\par\textbullet\,\,Regular skills using LaTeX for preparing documents.}

\ecvtitle{Organizing and communication}{Semana da Astronomia 2009 Organizer and Speaker}
\ecvitem{}{Escola Secundária da Moita}
\ecvitem{}{\textbullet\,\,Opening speaker on the visits by many student belonging to classes from primary to high school, 2 to 3 times per day during 5 days, with 60-80 simultaneous attending people, from 6 year-olds to adults.\par
\textbullet\,\,Created a poster with the diferent phases of life of many types of stars and mounted a space-shuttle flight simulator.}

\ecvtitle{Driving licence}{B}

\ecvsection{Sports}
\ecvitem{1997 - 2012}{Practiced Karate since he was 6 years old until his first years of college, building his character with discipline. Tought him to be calm, analytical and to avoid conflict whenever possible.}

\newpage

\ecvsection{Awards}
\ecvitem{November 2012}{Concurso de Engenharia da ReCreativa, NFIST - 1st Prize}
\ecvitem{}{With a set of materials sourced by the organization, the groups had to build something different each year, revealing the particular challenge on the begining of the competition. In this edition, each group had to build a bridge suspended between 2 tables at a fixed distance. The team that built the bridge that could support most weight before collapsing would win.}

\ecvsection{Publications}

\bibliographystyle{plain}
\nobibliography{publications} % bib file name

\ecvitem{Pub1}{\bibentry{pub1}}
\ecvitem{Pub2}{\bibentry{pub2}}

\end{europasscv}
\end{document}