\documentclass[english,a4paper]{europasscv}
\ecvname{Rúben Marques}
\ecvaddress{Estrada dos Paulinos Corte Esteval CCI-1812}
\ecvtelephone{(+351) 212 898 212}
\ecvmobile{(+351) 934 962 739}
\ecvemail{rubenmarques91@gmail.com}
\ecvgithubpage{https://github.com/ras-marques}
\ecvlinkedinpage{https://www.linkedin.com/in/ruben-marques-6179aa60/}
\ecvim{Skype}{ras.marques}
\ecvim{Discord}{Rúben Marques\#4748}
\ecvdateofbirth{16 June 1991}
\ecvnationality{Portuguese}
\ecvgender{Male}
\ecvpictureleft[width=3.8cm]{my_photo.png}

\usepackage[numbers]{natbib}
\usepackage{bibentry}
\usepackage{notoccite}

\begin{document}
\begin{europasscv}
\ecvpersonalinfo

\ecvsection[2.5mm]{Professional Experience} 

\ecvtitle[2.5mm]{April 2017 - Present}{Co-founder and CTO}
\ecvitem{}{SurgeonMate}
\ecvitem{}{Project started in the end of 2015, being full time in the pointed duration. This initiative aims to create a smartglasses device to record surgeries and the accompaning multimedia library software to catalog and store those recordings.
\par \textbullet\,\,Participation on planning, particularly on the technical side.
\par \textbullet\,\,Firmware development, particularly on Python using a Raspberry Pi, regarding the smartglasses hardware.
\par \textbullet\,\,PCB development using EAGLE6.5.0.
\par \textbullet\,\,CNC PCB milling.
\par \textbullet\,\,Software development using javascript, regarding the multimedia library.}

\ecvtitle[2.5mm]{Agosto 2015 - Abril 2017}{Suporte Informático}
\ecvitem{}{Instituto de Plasmas e Fusão Nuclear}
\ecvitem{}{Instituto Superior Técnico}
\ecvitem{}{\textbullet\,\,Assistência informática e administração de sistemas Linux e de redes informáticas.
\par \textbullet\,\,Capacidades regulares de programação em \textit{Python} e \textit{bashscript} na automação de tarefas.
\par \textbullet\,\,Capacidades regulares de construção de sites com \textit{html}, \textit{php} e \textit{mysql}. 
\par \textbullet\,\,Configuração de backups automáticos de servidores usando \textit{bacula}.
\par \textbullet\,\,Análise de rede e reconfiguração da \textit{firewall}.
\par \textbullet\,\,Reconfiguração de \textit{greylists} para administração automatizada de e-mails.}

\ecvtitle[2.5mm]{Setembro 2014 - Junho 2015\\Setembro 2013 - Janeiro 2014}{Monitor de atividades laboratoriais}
\ecvitem{}{Departamento de Física}
\ecvitem{}{Instituto Superior Técnico}
\ecvitem{}{Suporte teórico-prático a grupos de 16-18 alunos nas atividades laboratoriais a decorrer durante 4 horas por semana nas disciplinas de:
\par \textbullet\,\,Instrumentação Eletrónica (2014/2015)
\par \textbullet\,\,Sistemas de Aquisição de Dados (2014/2015)
\par \textbullet\,\,Microncontroladores (2013/2014)
\par Comunicação estuturada e direta em português, orientada à exposição de conceitos complexos para indivíduos com experiência básica ou inexistente na matéria.}

\ecvtitle{Novembro 2012 - Agosto 2013}{Bolseiro de Iniciação Científica}
\ecvitem{}{Instituto de Plasmas e Fusão Nuclear}
\ecvitem{}{Instituto Superior Técnico}
\ecvitem{}{Participação no projeto \textit{e-lab}: plataforma de experiências físicas reais disponíveis remotamente. Participação voluntária após final de Bolsa até Abril 2017.
\par \textbullet\,\,Desenvolvimento das montagens experimentais e sua respetiva instrumentalização e automação utilizando microcontroladores dsPIC e integração dessas experiências na plataforma utilizando linguagem \textit{Java} e \textit{XML}.
\par \textbullet\,\,2 co-autorias de publicações em \textit{Proceedings} numa conferência \cite{pub1} \cite{pub2} e um poster não publicado noutra.
\par \textbullet\,\,Experiência regular em comunicações em inglês, obtida em apresentações e visitas guiadas aos laboratórios remotos com grupos de 20-22 alunos internacionais. Participou nestas apresentações 2 vezes por ano entre Março 2012 e Março 2016.}

\ecvsection{Educação}
%[Add separate entries for each course. Start from the most recent.]

\ecvtitle{Setembro 2012 - Julho 2016}{MSc in Engeneering Physics and Technology}
\ecvitem{}{Instituto Superior Técnico de Lisboa, Portugal}
\ecvitem{}{15/20}
\ecvtitle{Setembro 2009 - Julho 2015}{Licenciatura em Engenharia Física e Tecnológica}
\ecvitem{}{Instituto Superior Técnico de Lisboa, Portugal}
\ecvitem{}{13/20}
\ecvtitle{Setembro 2006 - Julho 2009}{Ensino Secundário}
\ecvitem{}{Escola Secundária da Moita, Portugal}
\ecvitem{}{18/20}

\ecvsection{Aptidões e competências pessoais}

\ecvmothertongue[20pt]{Português}
\ecvlanguageheader
\ecvlanguage{Inglês}{C2}{C2}{C2}{C2}{C2}
\ecvlanguagefooter[10pt]

\newpage

\ecvitem{Atividades variadas com\\eletrónica e programação}{A sua Tese de Mestrado consistiu no desenho e prototipagem de um inversor eletrónico para o controlo de motores de indução do tipo Dahlander no âmbito dos veículos elétricos. Para além da Tese de Mestrado e das aulas leccionadas, o candidato ingressou em outras atividades que contribuiram para o desenvolvimento de conhecimentos de eletrónica e programação:\par
\textbullet\,\,Estudou o funcionamento, dimensionou e construiu uma bobina de Tesla no âmbito da disciplina de Introdução à Investigação no ano letivo 2013/2014.\par
\textbullet\,\,Construiu e programou de raiz uma fresadora CNC de 3 eixos nos tempos livres para utilização em projetos futuros, continuando os seus melhoramentos a nível de interface com o utilizador.
\par\textbullet\,\,Desde Junho 2014 faz parte da equipa de desenvolvimento de uma bomba de asma inteligente no âmbito da Tese de Mestrado da aluna Joana Belo 	
da Faculdade de Arquitectura de Lisboa. Está responsável pela escolha de sensores adequados e participa nas decisões estruturais funcionais do protótipo.
\par\textbullet\,\,Desde Setembro 2015 faz parte da equipa de desenvolvimento do projeto \href{http://www.surgeonmate.com/}{SurgeonMate}. Trabalha com \textit{raspberry pi}, câmeras, microfones, motorização e interfaces gráficas.
\par \textbullet\,\,De Novembro 2015 a Março 2016 fez parte do projeto de estudantes Técnico Solar Boat, onde desenvolveu um programa em MATLAB para simular a utilização de energia por um barco elétrico solar numa pista com parâmetros configuráveis.
\par \vspace{3mm}
Estas atividades, juntamente com a Tese de Mestrado e o apoio ao ensino resultaram em:\par
\textbullet\,\,+1000h de programação C de microcontroladores PIC
\par\textbullet\,\,$\sim$300h de programação C de microcontroladores Arduino
\par\textbullet\,\,$\sim$300h de desenvolvimento de circuitos de controlo e proteção de eletrónica de potência.
\par\textbullet\,\,$\sim$200h de desenvolvimento de circuitos de amplificação e filtragem de sinais analógicos e digitais.
\par\textbullet\,\,Utilização regular de ORCAD 9.2 e
EAGLE 7.2.0 no desenho de esquemáticos
e \textit{layouts} de Placas de Circuito Impresso.
\par\textbullet\,\,Utilização regular de MATLAB em simulações físicas.
\par\textbullet\,\,Utilização regular de LaTeX na elaboração de documentos.}

\ecvtitle{Organização e comunicação}{Semana da Astronomia 2009 Organizador e Palestrante}
\ecvitem{}{Escola Secundária da Moita}
\ecvitem{}{\textbullet\,\,Foi o orador de abertura das visitas das várias turmas do Ensino Primário ao Secundário, 2 ou 3 vezes por dia durante 5 dias, com uma assistência de 60 a 80 pessoas, desde crianças com 6 anos a adultos.\par
\textbullet\,\,Criou um póster das diferentes fases de vida de vários tipos de estrelas e organizou os recursos e a montagem de um simulador de voo de um vaivém espacial.}

\ecvtitle{Carta de condução}{B}

\ecvsection{Desporto}
\ecvitem{1997 - 2012}{Praticou Karate desde os 6 anos de idade até aos primeiros anos da licenciatura, contribuindo para a sua educação com disciplina. Ensinou-o a ser calmo, analítico e não conflituoso.}

\ecvsection{Prémios}
\ecvitem{Novembro 2012}{Concurso de Engenharia da ReCreativa, NFIST - 1º Prémio}
\ecvitem{}{Com um conjunto de materiais fornecidos pela organização, os grupos têm que construir algo diferente todos os anos, revelado apenas no início da competição. Nesta edição cada grupo teve de construir uma ponte suspensa entre 2 mesas a uma distância fixa e ganhava a que conseguisse suportar mais peso.}

\newpage

\ecvsection{Publicações}

\bibliographystyle{plain}
\nobibliography{publications} % bib file name

\ecvitem{Pub1}{\bibentry{pub1}}
\ecvitem{Pub2}{\bibentry{pub2}}

\end{europasscv}
\end{document}